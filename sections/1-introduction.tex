\section{Introduction}
When the World Wide Web was introduced in 1990, users identified themselves with user names and passwords, creating a new account for every service. Even though Single Sign-On has reduced the number of passwords per user, passwords are still a major security risk. In 2017, the password manager LastPass analysed the data of employees of over 30.000 companies using the service and found that the average amount of accounts per employee is 191 \cite{lastpass}. This is because identity storage is still centralized. If one wants to login to a service, the username and password are stored in a database owned by the service. 

The main disadvantage of this approach is that the service controls the users' data. As an example, the terms of service of Instagram state the following\footnote{\href{https://help.instagram.com/430517971668717}{Instagram's terms of service 2021}}: "We reserve the right to modify or terminate the Service or your access to the Service for any reason, without notice, at any time, and without liability to you". \cite{sovrin} clearly explains how this might impact end-users: "Because the only online identities most people have are centralized, the removal or deletion of an account effectively erases a person’s online identity which they may have spent years cultivating and may be of significant value to them, and impossible to replace." In addition, these data duplicates ensure that the estimated accumulated cost of identity assurance in the UK exceeds 3.3 billion pounds. CTRL-Shift has estimated that using 'make once, use many times' strategies could reduce this to 150 million pounds \cite{ctrl-shift}.

Self-sovereign identity aims to solve the problem by providing users with complete control over their data. This is achieved with decentralized data management, such as blockchain. In this context, decentralized means user-centric; the user is the only person storing and managing their data. The TrustChain SuperApp \cite{superapp} is a mobile application under development by the Delft Blockchain Lab. It aims to create a digital foundational identity. However, it currently cannot transfer data to other applications. This is an essential aspect of SSI to ensure third parties, such as the government, can request data from a user to confirm their identity.

This research will focus on creating a secure and reliable way to transfer data from the SuperApp to a third party. A possible use case for this is buying alcohol online. The SuperApp could be used to confirm that the buyer is actually of legal drinking age. There are some challenges to transferring data outside of the blockchain. These will be explored first in the Problem Description, then the chosen solution will be explained in Section \ref{solutions}.

Afterward, the possibilities for the communication protocol will be evaluated and discussed. The best one will be implemented and discussed in the section about the engineering contribution, where also the design will be explained. Then, I'll reflect on the ethical aspects of my research and a reflection on the results will be given in the discussion. Finally, the conclusion will contain a brief summary of the problem and solution and elaborate on future research that might be conducted in this field.

\textcolor{red}{Briefly explain my contributions}