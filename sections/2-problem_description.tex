\section{Problem Description}
Many SSI applications either aim to replace all current technology or do not have a solution for the data duplication that occurs in the databases of services like social media. They mainly focus on verifying the possession of certain documents, such as a driver's license. To reach a fully decentralized way of managing and storing data, no service should be allowed to store data regarding their users. They should rather request the data from the users themselves. 

The SuperApp currently does not support the transfer of data across applications. Thus the online identity that a user assembles and stores can only be used within the application itself. To define Self-Sovereign Identity, the ten principles that were devised by Christopher Allen are often used. The sixth of which is Data Portability: "Information and services about identity must be transportable" \cite[p. 14]{principles}. Interoperability is a key feature of SSI.

The current situation is not desirable as it implies that each application currently in use by end users would have to be replaced with an equivalent in the SuperApp. As mentioned previously, the average employee has 191 accounts across various platforms. The SuperApp has been designed to be able to replace most, if not all, of these. Still, it would be more effortless, both for users and developers, to make the SuperApp collaborate with other applications, rather than making it replace them. 

Naturally, one of the complications of transferring data out of the blockchain is security. Data could be intercepted or possibly even altered by a malicious user, who could reveal the data to anyone. SSI applications do try to solve this problem by using verifiable claims, which will be explored in section \ref{solutions}. Afterward, the communication protocol that will be used to send data to other applications will be described. 
