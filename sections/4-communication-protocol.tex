\section{Our claim portability framework}
In order to profit from data deduplication, we need to radically rethink the interoperability aspects of SSI. Most SSI solutions either assume that the user will only operate within their designated app or they do not mention how large corporations should incorporate the SSI solution in their service. Therefore, we will devise a universal claim portability framework for the sharing of verifiable claims. This communication protocol will be designed for general blockchain SSI solutions. When the design of the protocol has been finished, it will be implemented in the SuperApp to evaluate the usability. 

Within the SuperApp, users can already verify they are over eighteen and other users can attest to this. In this case, the issuer and subject both use the SuperApp, similar to other SSI implementations. The issuer is a random user in the blockchain in this case, there is no way to indicate a trusted party, such as a government. However, when designing this protocol it should be kept in mind that the relying party may not always be using the SSI solution. It is of utmost importance to ensure that third parties can communicate with the SuperApp to obtain information about their users, as SSI prevents them from storing it themselves. In this section, every aspect of the framework will be discussed, such as data storage and encryption.

\subsection{Data storage}
Identifying applications such as SSI solutions, are dealing with sensitive data. Since the GDPR law \footnote{\href{https://ec.europa.eu/info/law/law-topic/data-protection/reform/rights-citizens/my-rights/what-are-my-rights_nl}{European Commission: GDPR}} went into effect in 2016, users have the right to request, delete or alter their personal data that is stored at an organisation. This law has a substantial positive influence on the control users have over their data, but it is slightly complicated for organisations as it does require support from the framework to adequately provide the data to the right user. 

Where to store the personal user information is an important decision that has to be made when designing an SSI protocol. Blockchains are distributed and secure, but the information on the chain is visible to all peers. If there is a lot of information on the chain, the look-up time can be very high. Therefore, it is not suitable to store all information on-chain.  


\subsection{Content}
The most important decision for the design of the protocol is the decision of which data gets sent over the network. It is in the subject´s interest to send as little data as possible. It is always good to keep in mind that malicious users could get hold of the data. 
In section \ref{solutions} it was already mentioned that an attestation includes some metadata. This metadata should at least contain a validity period or expiration time of the transaction. This makes sure that malicious users cannot take advantage of unused or lost claims. This validity period should depend on the average response time. Next to that, a identifier or name of the transaction should also be included. If there are multiple transactions in progress between a subject and relying party, the identifier will make sure data does not get mixed up and it is clear which VCs are confirmed and which are denied. 

Last but not least it is important to know if a claim has to be attested by a trusted issuer. This could be the case when buying alcohol: The government should have attested that fact that the subject is of legal drinking age. However, some claims might be attested by the user themselves. An example of this could be accessing the website of the liquor store. For this, identification by the government is not necessary, but the user has to testify that they are actually of age. Currently, this happens by pressing a button to verify you are old enough. Using the SuperApp would actually store the data that you visited the website. This could be used for legality purposes. 

\subsection{Encryption}
The VCs cannot be sent in plaintext. That would make it very easy for malicious users to use the data in case they obtained it. Before sending, the VCs are encrypted with a key. The public keys of the user and the service are the identifiers that were mentioned before. These are stored on the blockchain. 

To give the user full control over their identity and keep the solution decentralized, the private keys should be stored on the user's device, which usually is a smartphone. The smartphone is portable and widely used. In 2018, 84\% of the Dutch citizens had access to a smartphone with internet connection \cite{eurostat}. This poses some threats of loss of keys upon losing the phone, for which there are some solutions to retrieve data. However, the problem of data resilience is out of the scope of this research and will not be discussed any further.

To encrypt a claim from a relying party to the subject, it is first signed by the relying party. It is encrypted with their private key, such that it can only be decrypted with their public key. This way, anyone could be able to decrypt the claim and verify it where it came from. To make sure only the intended receiver can decrypt the claim, it is also encrypted with the public key of the subject, such that it can only be decrypted with their private key, which is only known to the subject itself. The same principle is used for encrypting the answer to the claim. 

The advantage of private-public key pairs is that they are self-authenticating, they do not require a third trusted party to assign or verify the keys as opposed to, for example, Universally Unique Identifiers \cite{survey}. This strengthens the decentralized aspect of SSI as you do not rely on a third party to verify your identifier.

\subsection{Usability}
Many research papers and white papers about SSI explain the difference between centralized identity, federated identity and SSI. However, they do not explain concretely how they will replace popular applications such as social media. They mainly focus on verifying the fact that a user has a certain document, such as a driver's license. They seem to imply that all applications will eventually make use of the SSI solution, but do not describe how this switch from centralized/federated to full SSI will take place. 91\% of users of the internet know it is unsafe to reuse passwords, but 61\% still admit to doing exactly that. If users think it is too difficult to remember passwords, one can only imagine how thrilled they would be to replace all their daily applications just because SSI is more secure. Maybe some services will incorporate the SSI solution in their application, but only if it is easy to do so. 

\subsubsection{Developer usability}
For developers it is paramount that the SSI application is easy to integrate in their own application. To make decisions on this, we will take a look at other successful services that have been integrated in applications. Good examples of these are payment methods such as iDeal or PayPal. They have created API's that other applications can integrate such that they can communicate with the payment provider. 

Some organisations might use the SuperApp because it offers security. However, other will not and need extra persuasion. It is important to note that using the SuperApp will eventually remove the need for local data storage. Especially for small businesses, this can save a lot of money. The amount of data that is stored worldwide is growing rapidly, so all this data duplication that currently exists will no longer be feasible in the future. 

\subsubsection{User usability}
\cite{usability} has defined the main necessities to create an application that users can easily utilize. These attributes are learnability, efficiency, user retention over time, error rate and satisfaction. In short, the learning-curve should not be too steep, the process of using the application should be efficient and a user should not be able to make many mistakes. In the framework designed for this research that means that the actions should be explained clearly and concisely and the amount of actions the user should take should be limited. 

\textbf{\textcolor{red}{This is where my writing ends and the template begins}}
% https://www.tno.nl/nl/aandachtsgebieden/informatie-communicatie-technologie/roadmaps/data-sharing/ssi/ https://sovrin.org/wp-content/uploads/2018/03/Sovrin-Protocol-and-Token-White-Paper.pdf 
